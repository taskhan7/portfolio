      The CS51 Final Project consisted of implementing a small subset of the Ocaml language, called Miniml. 
This language consisted of a type expression that encompassed different atomic types such as integers and bools
and other types of constructs available in the Ocaml language, such as functions, conditionals, let statements, 
and even definitions of recursive functions. The Miniml language evaluates expressions in a way that incorporates 
substitution semantics, which is a method for substituting certain expressions for free variables in another
expression. It can also evaluate expressions in an dynamically-scoped environment. As a part of the project, we 
had to incorporate our own extensions to try and extend the Miniml language. The extensions that I incorporated 
consisted of adding more functionality to the language with the addition of two different operators, and 
implementing an additional form of evaluation that evaluates expressions in a lexically-scoped environment. 
      The two operators I added to the Miniml language were a GreaterThan operator and a Divide operator. This would 
let you divide two integers in the Miniml language and then compare either integers or bools with the greater than 
symbol to output a bool. For example, the expression "Binop(Divide, Num(5), Num(5))" would output an expression of Num(1). 
Similarily, an expression of "Binop(GreaterThan, Num(5), Num(5))" would output an expression of Bool(false) because 5 is 
not greater than 5. I implemented this by tweaking the type definition of an expression and then adding different match 
cases to my evaluation functions. I also had to add to the parser to make sure the REPL for the Miniml language worked 
to encompass this added functionality. 
      My second extension was a little trickier to implement. Dynamically-scoped semantics refers to ways of evaluating 
expressions, specifically functions, in the environment in which the function is applied while lexically-scoped semantics 
refers to ways of evaluating functions in the environment in which the function is defined. The two evaluation functions 
for a lexical versus a dynamic scope were pretty similar, but the lexical evaluation function differed for evaluations of 
functions, function applications, and recursive let statements. For an evaluation of a function in a lexically-scoped 
environment, you want to return a closure, which returns the function along with the environment in which the function was 
defined. Then, during function application, you evaluate it within the environment returned by the closure. For a recursive 
let, the process is more complex. You first need to evaluate the definition of the let in an environment where the variable 
maps to Unassigned, and then evaluate the body of the let in that environment,where instead of Unassigned, the variable maps 
to whatever result came from evaluating the definition. Using the functions from the environment module helped create this 
lexical evaluation function. So for an expression like "let x = 1 in let f = x + 5 in let x = 2 in f", the lexical evaluation 
would spit out 6 while the dynamic evaluation would spit out 7. Implementing these extensions helped improve my understanding 
of semantics in general, and gave me a better sense of what it takes to implement a functionality of a language. 
